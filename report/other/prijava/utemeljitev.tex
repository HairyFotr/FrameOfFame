\renewcommand{\thesubsection}{\arabic{subsection}}

Vedno več sodobnih umetniških del temelji na računalniški tehnologiji~\cite{digitalartconservation}. Tu se pojavi vprašanje: ``Kako ohraniti ta umetniška dela kljub hitrim spremembam računalniške tehnologije?''.

Tehnologija namreč napreduje in se razvija v vsakem trenutku, vedno več je novega, boljšega in uporabnejšega. Posledično se tehnologija, ki je lahko v nekem trenutku zelo aktualna, čez nekaj časa ne uporablja in posodablja več. To pa pomeni, da bodo dela, ki so bila zasnovana na nekoč novi, za danes pa že stari tehnologiji, potonila v pozabo. Tehnologijo, ki je takrat podpirala delovanje neke stvaritve, bo zamenjala nova, dela pa bodo tako pozabljena. To pa je stvar, ki jo želimo preprečiti. Naš namen bo umetnine pustiti točno take kot so, spremenili pa bomo njihovo ozadje. Tako povprečnemu uporabniku spremembe ne bodo vidne, umetniško delo bo povsem enako kot je bilo, vendar pa bo dostopno tudi z novejšo programsko opremo, na novejših napravah in na raznih socialnih omrežjih, kot je aktualno v današnjih časih. Dodali bomo tudi nekaj novih funkcionalnosti, ki bodo povečale prijaznost uporabniku in povečale njegovo zanimanje za delo. S tem, ko bo omogočena uporaba novejših platform, bo delo dostopno širšemu krogu ljudi, saj so le te vedno bolj dostopne, cenejše, pa tudi prenosljive~\cite{vzdrzevanjeProgramskeOpreme}.

Uspešnost projekta bomo ocenili tako, da bodo uporabniki lahko direktno primerjali staro in novo verzijo instalacije.
Na podlagi izku\v senj pri vzdr\v zevanju konkretne instalacije pa bomo sku\v sali prispevati k splo\v snim smernicam za vzdr\v zevanje ra\v cunalni\v sko zasnovanih
umetni\v skih instalacij.


\subsection{Enak zunanji izgled}

Z umetni\v sko zgodovinarskega vidika je pri vzdr\v zevanju ra\v cunalni\v sko zasnovane umetni\v ske instalacije zelo pomembno, da se ohrani
enak zunanji videz in enak uporabni\v ski vmesnik \cite{digitalartconservation} .Z vidika vzdr\v zevanja programske opreme pa je potrebno izbrati najbolj racionalno re\v sitev
pri selitvi na novo platformo.
\v Ce gre za selitev na novo strojno in programsko platformo, je velikokrat potrebno programsko opremo napisati na novo. Odvisno od obse\v znosti programske
kode in od vrste programskega jezika, se lahko odlo\v cimo v\v casih tudi za ponovno uporabo kode ali za kodiranje povsem na novo \cite{vzdrzevanjeProgramskeOpreme}.

Ker pa po drugi strani uporabniki pri\v cakujejo, da se programska oprema stalno posodablja in sledi novim trendom, bomo izdelali poleg osnovne verzije,
ki  po videzu in funkcionalnosti ne bo ni\v c druga\v cna od prvotne, \v se eno  novo verzijo
instalacije, ki bo bolj dinami\v cna in jo bo mo\v zno povezovati s socialnimi omre\v zji, ki jih ob nastanku instalacije \v se sploh ni bilo.

Kar  se tiče tehnologije, si želimo biti v koraku s časom, zato smo se odločili prenosni računalnik zamenjati s pametnim telefonom. Glavni prednosti le tega sta njegova majhnost in razširjenost med uporabniki. Omejili se bomo samo na uporabo pametnega telefona, saj podpira visoko ločljivost in možnost prenosa slike v živo (HDMI, Wi-Fi,...), in je tako najprimernejši za naš projekt.


\subsubsection{Pametni telefon (strojna oprema)}
Pred desetimi leti, ko je bil ustvarjen projekt  ``15 sekund slave'', so med uporabniki prevladovali osebni računalniki. Danes pa jih vztrajno izpodrivajo pametni telefoni, statistični podatki celo kažejo, da njihova prodaja še vedno narašča, saj so posameznikom vedno bolj dostopni.

Ti telefoni so močnejši in bolj funkcionalni kot računalniki izpred desetih let. Ena izmed najpomembnejših pridobitev je njihova majhnost, kar močno olajša postavljanje same instalacije v tako imenovani razstavni prostor. Poleg tega imajo današnji pametni telefoni večino strojne opreme, ki jo za ta projekt potrebujemo:
\begin{itemize}
\item Digitalni fotoaparat za slikanje obrazov.
\item Možnost prenosa visoko-ločljive slike na zunanji zaslon.
\item Dostop do interneta preko WIFI ali pa kar preko 3G/4G.
\item Dovolj hiter procesor.
\end{itemize}

Kot že omenjeno, si danes velika večina ljudi lasti svoj pametni telefon. Aplikacijo si bodo lahko namestili na svojo napravo in instalacijo simulirali kadarkoli in kjerkoli bodo želeli.


\subsubsection{Programska oprema}
Tudi pri izgledu same aplikacije ne sme biti veliko sprememb. Glavna ideja mora ostati ista: prikaz slike osebe, ki jo kamera zazna, slika pa je sestavljena iz slik drugih obrazov. Statično sliko, ki je bila prisotna pri prvotnem projektu, bomo zamenjali z animacijo v stilu ``igra 15'', ki bo uporabnika bolj pritegnila, da bo pred okvirjem počakal potrebnih 15 sekund.

Izvirno idejo, da so si lahko udeleženci prepisali ID slike in si jo nato preko spleta ogledali, bi lahko nadomestili s QR kodo. Obstaja tudi možnost povezovanja z socialnimi omrežji, kot so Twitter, Facebook, Instagram in drugimi, ki so danes zelo priljubljeni. Primer: slikamo QR kodo in slika se prenese na Instagram, lahko tudi na lasten zid na Facebook-u, kjer bi bila slika prikazana kot različica danes zelo popularnih ``selfie-jev''. Možnosti je neskončno.


\subsection{Nadgrajevanje projekta}
Projekt bo odprtokodno dostopen na Github-u, kjer bodo tisti, ki ga želijo nadgraditi, to lahko brez težav tudi storili. Ena izmed možnosti nadgradnje je, da bi bil projekt enostaven za dodajanje novih vtičnikov in novih povezav. Lep primer je tudi dodajanje različnih socialnih omrežij in spreminjanje animacij ob izdelavi slike.


\subsection{Tehnologija za izdelavo}
Zbirka programov, kateri so potrebni za izdelavo magistrske naloge, se lahko po potrebi poveča. Vendar pa so trenutno nujno potrebni:
\begin{description}
\item[GIT] Potreben za hranjenje zgodovine sprememb projekta. Zelo velikega pomena za odprtokodne programe. 
\item[Android Studio] Vključuje SDK za Android, ter lepo oblikovan IDE za enostavno in hitro izdelavo Android aplikacij.
\item[Java] Jezik, potreben za Android.
\end{description}