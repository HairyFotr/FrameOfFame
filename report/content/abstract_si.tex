V okviru magistrske naloge je bila ponovno implementirana in nadgrajena interaktivna
umetniška instalacija ``15 sekund slave''. Motivacija za
instalacijo je umetniško delovanje ameriškega pop-art umetnika Andyja Warhola.
``15 sekund slave'' izgleda kot klasična slika, a je dejansko računalniški
zaslon, okvirjen kot umetniška slika. Nad zaslonom je v okviru vgrajen
digitalni fotoaparat, ki je povezan z računalnikom v ozadju. Vsakih petnajst
sekund fotoaparat slika obiskovalce galerije, ki stojijo pred sliko. Na sliki
računalniški program poišče vse obraze in nato naključno izbere enega izmed
njih. Ta obraz nato z grafičnimi filtri program obdela tako, da pridobi tako
imenovani ``pop-art'' videz z manjšim številom živih barv, ki spominjajo na
slike slavnih osebnosti, ki jih je iz fotografij delal Andy Warhol. Ker je
prvotna instalacija nastala pred več kot deset leti in se je strojna oprema v
tem času že zelo spremenila, se je pokazala potreba po prilagoditvi aplikacije
novemu stanju tehnologije.
