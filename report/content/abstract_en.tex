This thesis is a reimplementation of already known art installation named
``15 sekund slave''~\cite{leonardo}. The first motivation of installation was
movement of American famous artist Andy Warhol. His idea of pop-art, to change
a simple portrait, item or any picture into real art picture. And so it is
with ``15 sekund slave''. It looks like simple classic picture on the wall.
But in the background it is more then that. What public see is a monitor
combined with wooden art-look frame. On the top, there is small camera which
is connected to computer hidden behind the scene. Every fifteen seconds, the
camera take a picture of the place and send it to computer, where all the
magic happens. It search for human faces on the picture, pick one and apply a
combination of image filters which convert real life picture into a ``pop-art''
picture. The idea is still great, even ten years later, but the technology
used back there gets older and older. So we want to keep the idea but use
technology of current time to keep the art alive~\cite{trifonova}.
