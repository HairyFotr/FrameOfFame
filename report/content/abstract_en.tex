This thesis describes the reimplementation of the interactive art installation
entitled ``15 secunds of fame''. The motivation for the installation was the
famous American artist Andy Warhol. He modified simple photograph portraits
into pop-art pieces. The production of such portraits is the goal of the
installation ``15 secunds of fame''. The installation looks like a picture on
the wall, but it is in fact a computer monitor framed as picture. On the top
of the frame there is small camera which is connected to a computer hidden
behind the scene. Every fifteen seconds, the camera takes a picture of the
scene in front and sends it to the computer, where all the magic happens. It
searches for human faces on the picture, picks one face randomly and applies a
combination of image filters which converts the face into a ``pop-art''
portrait. The idea is still great, even ten years later, but the original
technology is dated and difficult to maintain more and more. We reimplemented
the installation with new hardware and software and we discuss the problems of
maintaining new media art installations in general.
